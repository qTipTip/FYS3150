\documentclass[intlimits]{amsart}
\usepackage[]{xcolor} 
\usepackage[]{graphicx} 
\usepackage[]{hyperref}
\usepackage[]{cleveref}

\renewcommand{\v}[1]{\mathbf{#1}}
\newcommand{\intlimits}{\int_{0}^\infty\int_0^\infty\int_0^{\pi}\int_0^\pi\int_0^{2\pi}\int_0^{2\pi}}
\newcommand*{\github}[1]{\def\@github{\color{gray!70}#1}}
\newcommand{\githublogo}{\raisebox{-1pt}{\includegraphics[height=9pt]{github}}}

\setcounter{tocdepth}{1}
\title{Electrons in a helium atom}
\author{Ivar Haugal\o kken Stangeby}
\begin{document}
\begin{abstract}
  In this project we wish to compute the expected value of the ground state
  correlation energy between two electrons in a helium atom. In doing so, we also
  take the opportunity to discuss various methods of numerical integration.
  Namely Gaussian quadrature with both Laguerre and Legendre polynomials; brute
  force Monte Carlo integration, and using importance sampling.\\
  \noindent
  All source code can be found on my GitHub page: \\

  \centering{\href{https://github.com/qTipTip/FYS3150}{\githublogo \, \color{gray!50}\textit{github.com/qTipTip}}}
\end{abstract}

\maketitle

\begin{figure}[h!]
  \centering
  \includegraphics[width=0.8\linewidth]{ch.pdf}
\end{figure}


\tableofcontents
\section{Introduction}
\label{sec:introduction}
We define the position of electron $i$ as $\v{r}_i = x_i\v{e}_x + y_i\v{e}_y +
z_i\v{e}_z$, and define $r_i = \sqrt{x_i^2 + y_i^2 + z_i^2}$.  The \emph{ground
state correlation energy} of two electrons --- and more specifically its
expectation value is given by the integral
\begin{equation}
  \label{eq:cart}
  \langle \frac{1}{\left| \v{r}_1 - \v{r}_2 \right|} \rangle = \int_{-\infty}^{\infty} d\v{r}_1d\v{r}_2 e^{-2\alpha(r_1, r_2)} \frac{1}{\left| \v{r}_1 - \v{r}_2 \right|}.
\end{equation}
In this project we fix $\alpha=2$ which corresponds to the charge of the helium
atom. It can be shown that this specific integral has a closed form solution of
$5\pi^2/16^2$ which we will use to assess the numerical stability and
effectiveness of our numerical methods.

We will later, when we apply the Gauss-Laguerre quadrature, need our integral
rewritten in spherical coordinates. The differentials are then given by
\begin{equation}
  \label{eq:diff}
  d\v{r}_1d\v{r}_2 = r_1^2dr_1r_2^2dr_2\sin\theta_1d\theta_1\sin\theta_2d\theta_2d\phi_1d\phi_2,
\end{equation}
with a scale factor of
\begin{equation}
  \label{eq:scale}
  \gamma(\beta) = \frac{1}{r_{12}} = \left( r_1^2 + r_2^2 - 2r_1r_2\cos\beta \right)^{-1/2}
\end{equation}
where $$\cos\beta = \cos\theta_1\cos\theta_2 +
\sin\theta_1\sin\theta_2\cos(\phi_1 - \phi_2).$$

\section{Gaussian Quadrature}
\label{sec:gaussian_quadrature}

In Gaussian Quadrature, the main idea is to rewrite an integral as a weighted
sum of function evaluations,
\begin{equation}
  \notag
  I = \int_a^b f(x) \, dx = \sum_{i=1}^{N} \omega_i f(x_i)
\end{equation}
however, we can do use various orthogonal polynomials which arise from
different differential equations in the natural sciences. We will make use of
both Legendre and Laguerre polynomials. The only downfall of these polynomials
are that their orthogonality relation is only satisfied on certain intervals,
hence we need to pick our polynomials carefully.

\subsection{Legendre}
\label{sub:legendre_polynomials}

The Legendre polynomials are defined on the interval $[-1, 1]$ with a weight
function $W(x) = 1$.  This would lead us to believe that we can only perform
integrals over the interval $[-1, 1]$, but a simple change of variables lets us
use any arbitrary interval. However, one must note that this can significantly
degrade the numerical accuracy. The Gauss-Legendre quadrature gives rise to the following
relation:
\begin{equation}
  \notag
  I = \int_{-1}^{1} f(x) \, dx = \int_{-1}^{1}W(x)g(x)\, dx = \int_{-1}^{1}g(x)
  \, dx = \sum^{N}_{i=1} \omega_ig(x_i).
\end{equation}
It turns out that the associated integration points $x_i$ are the roots of the
Legendre polynomials. We can rewrite \cref{eq:cart} in the Legendre form using a simple change of variables.

\subsection{Laguerre}
\label{sub:laguerre_polynomials}

The Laguerre polynomials are orthogonal on $[0, \infty)$ and have an associated
weight function $W(x) = x^\alpha e^{-x}$. Again, the orthogonality property gives us the relation
\begin{equation}
  \notag I = \int_0^\infty f(x) \, dx = \int_0^\infty W(x)g(x) \, dx =
  \int_0^\infty x^\alpha e^{-x} g(x) \, dx = \sum^{N}_{i=1} \omega_ig(x_i).
\end{equation}

In order to use the Gauss-Laguerre quadrature, we need to rewrite
\cref{eq:cart} to spherical coordinates. We do this using \cref{eq:diff} and
\cref{eq:scale}. This gives us
\begin{align*}
  I &= \int_{-\infty}^{\infty} d\v{r}_1d\v{r}_2 e^{-4(r_1, r_2)} \frac{1}{\left| \v{r}_1 - \v{r}_2 \right|}.\\
    &= \intlimits r_1^2r_2^2e^{-4r_1} e^{-4r_2} \gamma(\beta)\sin\theta_1\sin\theta_2 dr_1dr_2d\theta_1d\theta_2d\phi_1d\phi_2\\
  \intertext{Substituting $u_i = 4r_i$, $i = 1, 2$ yields}
  &= \frac{1}{1024}\intlimits u_1^2u_2^2e^{-u_1}e^{-u_2} \frac{\sin\theta_1\sin\theta_2}{\sqrt{u_1^2 + u_2^2 - 2u_1r_2\cos\beta}} du_1du_2d\theta_1d\theta_2d\phi_1d\phi_2\\
    &=\frac{1}{1024} \int_{0}^\infty\int_0^\infty\int_0^{\pi}\int_0^\pi\int_0^{2\pi}\int_0^{2\pi}W(u_1)W(u_2)g(u, \theta, \phi)\,du_1du_2d\theta_1d\theta_2d\phi_1d\phi_2.
\end{align*}

Based on these integral limits it can be wise to apply Gauss-Laguerre only for
the first two integrals, and use Gauss-Legendre for the other four.

\section{Monte Carlo Integration}
\label{sec:monte_carlo_integration}

In Monte Carlo integration we use a \emph{non-deterministic} approach to
computing a definite integral. We sample integration points at random. The
naive approach is to just pick numbers at random in the interval $[0, 1]$,
which is what we will do in the brute force Monte Carlo method however, as we
will see, another method is to use so-called \emph{importance-sampling} to find
a probability distribution function (PDF) that closely resembles the function
we wish to integrate. Since these methods are non-deterministic, there is
always an error associated with the random aspect of the computations. We
therefore, in addition to compute the integral itself, compute the standard
deviation to get a grip on the uncertainty in our results. One of the effects
of importance sampling is to reduce the standard deviation.

\subsection{Brute force Monte Carlo}
\label{sub:brute_force_monte_carlo}

The probability distribution function associated with the brute force Monte Carlo method is
the \emph{uniform distribution} given by
\begin{equation}
  \notag
  p(x) = \begin{cases}
    1, & x \in [0, 1] \\
    0, & x \in [0, 1]^c.
  \end{cases}
\end{equation}
Since $p(x)$ satisfy $\int_{-\infty}^\infty p(x) \, dx = 1$ we have that
\begin{align*}
  \notag
  \langle f \rangle = \int_{-\infty}^\infty p(x) f(x) \, dx = \int_0^1 p(x)f(x) \, dx &\approx \sum^{N}_{i=1} f(x_i).
\end{align*}
In order to apply this for more general integration limits $a, b$ we simply
note that by conservation of probability $p(y)dy = dx$, hence standard change
of variables will suffice: $y(x) = a + (b - a)x$.

\subsection{Importance Sampling Monte Carlo}
\label{sub:importance_sampling_monte_carlo}

We now wish to find a probability distribution function $p(y)$ that closely
mimics the behavior of $f(x)$ over our chosen integration interval.
For our integral, it is suitable to use the exponential distribution $p(y) = e^{-y}$, thus
$p(r) = e^{-4r}$. However, in order to apply this, we must have the normalization criteria satisfied, hence
\begin{equation}
  \notag
  p(r) = Ae^{-4r}
\end{equation} 
is a suitable probability distribution function. Note that we
wish to express our integral in spherical coordinates in order to apply this
specific probability distribution.

In order to make this work for arbitrary intervals, we introduce the change of variable
\begin{equation}
  \notag
  y(x) = -\frac{\ln(1 - x)}{2\alpha}
\end{equation}
which will give us values $y \in [0, \infty)$.

\section{Implementation}
\label{sec:implementation}

In this project we employ a lot of \emph{black-box} functions, so there aren't
that many implementation details to talk about. However, I have included some
thoughts on selected methods.

\subsection{Gauss-Legendre}
\label{sub:gauss_legendre}

In order to evaluate the integral using Gaussian Quadrature with Legendre
polynomials, we need to find suitable integration limits. Clearly, we can't
integrate from $-\infty$ to $\infty$ numerically. We observe that the function
$e^{-4r_i}$ is \emph{essentially zero} for $r_i \approx 3$. This can be
verified by the following calculations. Say we want $e^{-4r_i} = 10^{-5}$:
\begin{equation}
  \notag
  e^{-4r_i} = 10^{-5} \Longrightarrow r_i = -\ln(10^{-5}) / 4
\end{equation}
which gives us that $r_i \approx 2.88$. We can therefore chose to neglect any
errors that arise from incorrect integration limits by integrating from $-3$ to
$3$ for instance.

\subsection{Gauss-Laguerre}
\label{sub:gauss_laguerre}

For the integral in spherical coordinates we chose to use Laguerre polynomials
only for the two radial integrals, and leave the integrals with respect to
$\theta$ and $\phi$ for Legendre polynomials which is more suited for such
narrow integration limits.

\section{Results}
\label{sec:results}

\subsection{Gauss-Legendre}
In \cref{tab:legendre} we see that the convergence rate of the Gauss-Legendre
method is quite poor and the numerical method can be seen as quite unstable.
Increasing the number of integration steps does not necessarily translate
directly into a more accurate approximation, as can be seen for the $N=25$ to
$N=30$ jump. We have in these simulations used the integration interval $[-3,
3]$.
\begin{table}[p]
  \centering
  \caption{In this table we present the computed integral, the absolute error
    in the calculations as well as the time spent for $N$ integration steps for the Gauss-Legendre
    quadrature. The time complexity of the integral itself is $\mathcal{O}(N^6)$
which is pretty immense, and can clearly be seen based on the increase in time
elapsed for increasing $N$. }
  \label{tab:legendre}
  \begin{tabular}{cccc}
    $N$ & Result & Absolute error & Time [sec]\\
    \hline
    \hline
    5& 0.2642& 0.0714& 0.0004\\
    10& 0.0719& 0.1207& 0.0267\\
    15& 0.2390& 0.0463& 0.3184\\
    20& 0.1561& 0.0366& 1.7411\\
    25& 0.1958& 0.0030& 6.6549\\
    30& 0.1772& 0.0154& 19.6983\\
    35& 0.1899& 0.0028&55.4347\\
    40&0.1844&0.0083&117.377\\
    \hline
  \end{tabular}
\end{table}

\subsection{Gauss-Laguerre}
\label{sub:gauss_laguerre}
As seen in \cref{tab:laguerre}, converting to spherical coordinates and
applying Gauss-Laguerre for the two integrals with limits $0$ and $\infty$
dramatically increases the numerical approximation. A strong convergence
towards the exact value is observed. A slight increase in elapsed time is seen,
however only $N = 15$ integration steps is required for a result that matches
the best achieved using only Gauss-Legendre.
\begin{table}[p]
  \centering
  \caption{Presented is the computed integral, the absolute error in calculations
  as well as time elapsed for $N$ integration steps. The time complexity of the
  integral itself is again $\mathcal{O}(N^6)$ however the numerical method is
  converging properly as opposed to the Legendre quadrature.}
  \label{tab:laguerre}
  \begin{tabular}{cccc}
    $N$ & Result & Absolute error & Time [sec]\\
    \hline
    \hline
    5& 0.1734& 0.0193& 0.0011\\
    10& 0.1864& 0.0063& 0.0675\\
    15& 0.1897& 0.0030& 0.8190\\
    20& 0.1910& 0.0016& 4.3892\\
    25& 0.1917& 0.0010& 16.8339\\
    30& 0.1921& 0.0006& 50.1823\\
    35& 0.1923& 0.0004& 128.2480\\
    40& 0.1924&0.002&291.6470\\
    \hline
  \end{tabular}
\end{table}

\subsection{Brute Force Monte Carlo}
\label{sub:brute_force_monte_carlo}

In \cref{tab:MC_brute} we see immediately that the Monte Carlo methods require
a very large number of integration steps. We also see that the results converge
quickly, and the standard deviation -- although fluctuating a fair bit -- also
decreases with increasing $N$. The time elapsed for this brute force method
mimics the time elapsed for the brute force Gauss-Legendre method.

\begin{table}[p]
  \centering
  \caption{Using brute force Monte Carlo integration we see that the results
  converge to the exact fairly quickly. The computed standard deviation however
fluctuates a fair bit. None the less, the brute force Monte Carlo method is
have similar performance to the brute force Gauss-Legendre integral.}
  \label{tab:MC_brute}
  \begin{tabular}{ccccc}
    N & Result & Absolute error & Standard Deviation & Time [sec]\\
    \hline\hline
    $10^3$ &0.0235 &0.1691 &0.0089 &0.00007\\
    $10^4$ &0.0473 &0.1453 &0.0200 &0.00073\\
    $10^5$ &0.1732 &0.0195 &0.0497 &0.00632\\
    $10^6$ &0.1759 &0.0168 &0.0159 &0.06715\\
    $10^7$ &0.1829 &0.0097 &0.0080 &0.64801\\
    $10^8$ &0.1981 &0.0053 &0.0030 &6.46146\\
    $10^9$ &0.1938 &0.0010 &0.0010 &64.5177\\
    \hline
  \end{tabular}
\end{table}

\end{document}
