\section{The Metropolis-Hastings Algorithm}

In this project we need random samples from a probability distribution
$P_\beta(\sigma)$ associated with the physical system we wish to model.
However, direct sampling can be difficult due to the fact that we need to
compute the partition function in order to express the probability distribution
in its entirety.

The Metropolis-Hastings algorithm is very well suited for this type of problem
because in order to sample from a specific distribution $P_\beta(\sigma)$ all
we need is a function $f$ proportional to the distribution density.

\subsection{The Metropolis Algorithm}
\label{sub:the_metropolis_algorithm}

Under the assumption that the proposed probability distribution function is
symmetric, we can make some simplifications to the Metropolis-Hastings
algorithm. This is what we call the Metropolis Algorithm.  The general
procedure can be described as follows:

\begin{enumerate}
  \item Initialization: We pick an initial sample.
  \item For every iteration: \\
    \begin{enumerate}
      \item We generate a new candidate for the new sample by picking a random element in the sample. 
      \item We calculate the acceptance ratio, which decides whether to reject
        or accept the new candidate. (This is why the Metropolis algorithm is
        good for this sort of problem. When computing the acceptance ratio, we
        divide out the normalization factor $Z$, which is the computationally
        heavy bit.)
      \item If the acceptance ratio is greater than one, we automatically
        accept the candidate. Otherwise, we accept the candidate with a
        probability equal to the acceptance ratio. In the case of a rejection
        the new sample is set to the previous one.
    \end{enumerate} 
\end{enumerate}
