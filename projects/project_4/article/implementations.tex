\section{Implementation}
\label{sec:implementation}

In this section we take a closer look at some of the implementation specific
details in this project. The Ising model was implemented in \textsc{C++} and
the class \texttt{Ising.cpp} contains all the methods needed for simulating the
model.

\subsection{The Ising class}
\label{sub:the_ising_class}
I chose an object-oriented approach to this implementation because I wanted the
possibility of having class instances that represented a physical system at any
given time. By wrapping all the methods in a class I could easily rerun a
similar simulation for a different temperature by re-initializing the system
and keeping any values from the previous temperature.\footnote{If I managed to
pull any of this through by any stretch of the imagination can be discussed.} I
also had a boolean flag for checking whether the system had been previously
thermalized for a previous temperature - in that case, one could skip the
thermalization.

I did not manage to figure out a way of outputting data that would satisfy the
demands of all the text-problems in the project at once, so I did have to do
some monkey-patching for some of them.

The computationally heavy methods are the \texttt{simulate} and
\texttt{metropolis}-methods, where \texttt{simulate} calls \texttt{metropolis}
once for each Monte-Carlo cycle. If we let $L$ denote the grid dimension and
$M$ the number of Monte-Carlo cycles, then each call of the \texttt{metropolis}
method constitutes a complexity of $\mathcal{O}(L^2)$. The \texttt{simulate}
method calls \texttt{metropolis} $M$ times, hence the total complexity for one
simulation for one temperature is $\mathcal{O}(ML^2)$.

\subsection{Main method}
\label{sub:main_method}
The main method is fairly simple, given a temperature range $[T_1, T_2]$ it
assigns a subinterval based on the number of nodes running in parallel to each
node. Each node then proceeds to simulate the system for its assigned
temperature range, and outputs its data the same data-file. The sorting of the
data is done by \texttt{Python}. The parallelization is done using OpenMPI.

\subsection{Errors}
\label{sub:errors}

My program gives correct values for certain systems, and very wonky values for
other systems. I have not been able to determine whether this is an artifact of
running too few simulations, whether my temperature resolution should have been
higher, or whether it is an artifact introduced by the stochastic nature of the
algorithms or (most probable) there is a bug somewhere in my program.

Based on the plots produced, all the weird behavior occurs in the quantities
directly dependent on the magnetization, which might indicate that that is
where the problem lies.
